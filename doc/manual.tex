\documentclass{book}
\usepackage[latin1]{inputenc}
\usepackage{english}
\usepackage[dvips]{graphicx}
\textwidth144mm
\textheight244mm
\title{LPTK}
\begin{document}
\title{LPTK}
\maketitle
\chapter[Basics]{Basics}
\section[Types]{Types}
Describes the basic types defined by the lptk.
\subsection[TAlignment]{TAlignment}
\textit{Declaration:} \texttt{\textbf{Type} TAlignment = (alCenter, alJustify, alLeft, alRight);}\newline
\textit{Values:}
\begin{itemize}
    \item alCenter
    \item alJustify
    \item alLeft
    \item alRight
\end{itemize}
\textit{Defined in:} gfxbase

\subsection[TOrientation]{TOrientation}
\textit{Declaration:} \texttt{\textbf{Type} TOrientation = (orHorizontal, orVertical);}\newline
\textit{Values:}
\begin{itemize}
    \item orHorizontal
    \item orVertical
\end{itemize}
\textit{Defined in:} gfxbase

\chapter[Classes and Objects]{Classes and Objects}
You can find a description of objects and classes come with the lptk-core package.

\section[TWidget]{TWidget}
TWidget is the base of all other visual lptk-widgets. Every widget is an descendant of TWidget.
\subsection[protected]{protected}
\begin{itemize}
    \item	\texttt{\textbf{procedure} SetDimensions(Left, Top, Width, Height : TgfxCoord);} 
		\newline Sets the position and size of the widget.
    \item 	\texttt{\textbf{procedure} MoveResizeBy(DeltaLeft, DeltaTop, DeltaWidth, DeltaHeight : TgfxCoord);} 
		\newline Moves and/or resizes the widget based of the differences given by the parameters.
\end{itemize}
\subsection[public]{public}
\subsection[events]{events}
\textit{Defined in:} gfxwidget
\end{document}
